\documentclass[a4paper, 12pt]{article}

\usepackage{url,comment}

\title{Classification of Dog Breeds using PCA}
\author{Dylan Wetherald\\Michael Gorlin\\Robert Shaffer}
\date{May 2015}

\begin{document}

\maketitle

\section*{Dataset}
Our dataset was a Stanford \footnote{\url{http://vision.stanford.edu/aditya86/ImageNetDogs/}}
provided file of dog images taken from ImageNet for the purpose of machine learning.
The dataset contains 120 different dog breeds with about 150 to 200 images per class. For
our project, the dataset was shrunk to 6 dog breeds | Toy Poodle, Afghan Hound, Australian
Terrier, Beagle, Boston Bulldog, and Siberian Husky | with only about 80 images per class. 
These breeds were chosen based on their distinct differences and slightly because of
personal significance. The criteria for valid images were: they contain only one dog, no
other animals are present, there was no text in the image, the image was decently sized, and 
there was contrast between the background and the dog.

The breakdown of our dataset was 60\% training data, 20\% Cross Validation, and 20\% Test data.
Over the course of learning these images are scaled to 100$\times$100 and converted to gray 
scale.
\section*{Method}
\section*{Images}
\section*{Results}
\section*{Conclusion}









\end{document}
